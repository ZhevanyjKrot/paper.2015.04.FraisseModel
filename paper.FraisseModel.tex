%%%%%%%%%%%%%%%%%%%%%%%% ????? %%%%%%%%%%%%%%%%%%%%%%%%%%%%%%%
\documentclass{article} % ????? ?????????
\usepackage[utf8]{inputenc}         % ??????? ?????????
\usepackage[russian]{babel}           % ???????????
\usepackage{amssymb,amsfonts,amsmath,eucal} % ????? AMSLaTeX
\usepackage{graphicx}                 % ??????? ???????
\usepackage{enumerate}                %
\usepackage{color}
\usepackage{hyperref}
\usepackage[colorinlistoftodos]{todonotes} % remove


%%%%%%%%%%%%%%%%%%%%% ????????? ???????? %%%%%%%%%%%%%%%%%%%%%
%%%%%%%%%%%%%%%%%%%%%%%%%%%%%%%%%%%%%%%%%%%%%%%%%%%%%%%%%%%%%%

\title{Ультраоднородные и экзистенциально замкнутые группы в классе абелевых групп}
\author{А.А. Мищенко, В.Н. Ремесленников, А.В. Трейер.}

\newtheorem{theorem}{Теорема}
\newtheorem{lemma}{Лемма}
\newtheorem{proposition}{Предложение}
\newtheorem{statement}{Утверждение}
\newtheorem{corollary}{Следствие}

\newtheorem{definition}{Определение}

\newcommand{\todoi}[1]{\todo[inline]{#1}}
\def\proof{{\noindent{\bf Доказательство.}} }
\def\A{{\mathfrak{A}}}
\def\K{{\mathcal{K}}}
\def\U{{\mathcal{U}}}
\def\P{{\mathcal{P}}}
\def\F{{\mathcal{F}}}
\def\S{{\mathcal{S}}}
\def\L{{\mathcal{L}}}
\def\C{{\mathcal{C}}}
\def\Z{{\mathbb{Z}}}
\def\N{{\mathbb{N}}}
\def\Q{{\mathbb{Q}}}
\def\Th{{\mathrm{Th}}}
\def\Tha{{\mathrm{Th}_\forall}}
\def\The{{\mathrm{Th}_\exist}}
\def\CG{{\mathrm{CGr}}}
\def\ui{{\mathrm{UI}}}
\def\HP{\textbf{HP}}
\def\JEP{\textbf{JEP}}
\def\AP{\textbf{AP}}
\def\AAP{\textbf{AAP}}


\begin{document}
\maketitle
\tableofcontents
\listoftodos



\section{Введение}

\section{Предел Фраиссе}

Пусть $\K$ некоторый универсальный класс. Определим некоторые свойства на классе $\K$ согласно книге Ходжеса [???] \todo{Вставить ссылку на Ходжеса}.

\noindent \HP  (\textit{Hereditary property}): Если $A \in \K$ и $B$ кончно порожденная подструктура в $A$, то $B$ изморфна некторой структуре в $\K$. 

\noindent \JEP  (\textit{Join Embeding Property}): Если $A, B \in \K$ тогда существует такая структура $C \in \K$, что $A$ и $B$ вкладываются в $C$.

\noindent \AP  (\textit{Amalgamation Property}): Если $A, B$ и $C$ структуры из $\K$ и $e : A \rightarrow B$ и $f : A \rightarrow C$ -- вложения, тогда существует структура $D \in \K$ и вложения $g : B \rightarrow D$ и $h : C \rightarrow D$ такие, что $ge = hf$. 

Пусть $G$ -- группа, \textit{$p$-высотой} элемента $g \in G$ называется наибольшее число $k$ такое, что в группе $G$ разрешимо уравнение $p^k x = g;$ $p$-высоту элемента $g$ будем обозначать $h_{p,G}(g).$ 

Введем новое свойство \AAP{} путем расширения свойства \AP{} следующим образом:

\noindent \AAP (\textit{Admissible Amalgamation Property}): Если $A, B$ и $C$ структуры из $\K$ и $e : A \rightarrow B$ и $f : A \rightarrow C$ -- вложения такие, что для любого элемента $a \in A$ $h_{p,B}(e(a)) = h_{p,C}(f(a))$. Тогда существует структура $D \in \K$ и вложения $g : B \rightarrow D$ и $h : C \rightarrow D$ такие, что $ge = hf$. 

Рассмотрим универсальный класс $\K$ порождающийся своей периодической частью, то есть $\K = ucl(T(\K))$, где $T(\K) = \{T(A) | \ A \in \K\}$. Пусть $C(\K)$ каноническая группа для класса $\K$. Рассмотрим класс всех конечно порожденных подгрупп группы $C(\K)$, обозначим его $FG(C(\K))$.

\begin{lemma}
Пусть класс $\K$ порождается своей периодической частью. Тогда для класса $FG(C(\K))$ выполняются свойства \HP, \JEP{} и \AAP.
\end{lemma}

\proof Выполнение свойства \HP{} следует из определения.

\todo{Вставить доказательство.}
% Рассмотрим случай, когда группа $C(\K)$ разлагается в прямую сумму нескольких копий примарных циклических групп одного порядка: $C(\K) = C^n(p^k)$.

% Пусть $A, B$ и $C$ группы из $FG(C(\K))$ и $e : A \rightarrow B$ и $f : A \rightarrow C$ -- вложения сохраняющие высоты, то есть для любого элемента $a \in A$ $h_p(e(a)) = h_p(f(a))$. Группы $A, B$ и $C$ раскладываются в прямую сумму примарных циклических $p$-групп 

Будем говорить, что группа $D$ обладает свойством \textit{$h_p$-ультраоднородности}, если любой гомоморфизм сохраняющий высоты между двумя конечными подгруппами $D$ расширяется до автоморфизма группы $D$.

\begin{theorem}[Аналог теоремы Фраиссе]
Пусть $\K$ универсальный класс для которого выполнены свойства \HP, \JEP{} и \AAP. Тогда существует единственная структура с точностью до изоморфизма $F_{hp}$ обладающая следующими свойствами:
\begin{enumerate}
\item $F_{hp}$ счетная;
\item Любая конечно порожденная группа из $\K$ является подгруппой в $F_{hp}$;
\item $F_{hp}$ обладает свойством $h_p$-ультраоднородности.
\end{enumerate}
\end{theorem}

\proof \todo{Вставить доказательство.}

\begin{theorem}
Если $\K$ универсальный класс для которого существует предел Фраиссе $F$, тогда для класса $\K$ существует и предел $F_{hp}$, такой, что $F \cong F_{hp}$.
\end{theorem}

\proof \todo{Вставить доказательство.}

\section{Экзистенциально замкнутые $\K$-группы}

Пусть $\K$ универсальный класс абелевых групп. 

\begin{definition}
Группа $A$ называется экзистенциально замкнутой относительно класса $\K$, если любая конечная система уравнений и неравенств из группы $A$ совместная с $\Tha(\K)$ имеет решение в группе $A$.
\end{definition}

Обозначим $\K_{EC}$ класс экзистенциально замкнутых групп из класса $\K$.

\todo{Вставить определение модельного компаньона.}
\begin{theorem}
Для любого универсального класса $\K$ из $\A_p$, класс $\K_{EC}$ аксиоматизируем, и следовательно, $\Tha(\K)$ имеет модельный компаньон ($\Th(\K_{EC})$).
\end{theorem}

\proof \todo{Доказать.}

 
\end{document}