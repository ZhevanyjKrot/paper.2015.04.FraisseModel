%%%%%%%%%%%%%%%%%%%%%%%% ????? %%%%%%%%%%%%%%%%%%%%%%%%%%%%%%%
\documentclass[14pt]{extarticle} % ????? ?????????
\usepackage[utf8]{inputenc}         % ??????? ?????????
\usepackage[russian]{babel}           % ???????????
\usepackage{amssymb,amsfonts,amsmath,eucal} % ????? AMSLaTeX
\usepackage{graphicx}                 % ??????? ???????
\usepackage{enumerate}                %
\usepackage{color}
\usepackage{hyperref}
\usepackage{anyfontsize}
\usepackage[colorinlistoftodos]{todonotes} % remove

\usepackage[
    left=25mm,
    top=25mm,
    right=45mm,
    bottom=25mm,
    footskip=10mm,
    nohead,
    marginparwidth=40mm,
    driver=xetex
]{geometry}


%%%%%%%%%%%%%%%%%%%%% ????????? ???????? %%%%%%%%%%%%%%%%%%%%%
%%%%%%%%%%%%%%%%%%%%%%%%%%%%%%%%%%%%%%%%%%%%%%%%%%%%%%%%%%%%%%

\title{Ультраоднородные и экзистенциально замкнутые группы в универсальных классах абелевых групп}
\author{А.А. Мищенко, В.Н. Ремесленников, А.В. Трейер.}

\newtheorem{theorem}{Теорема}[section]
\newtheorem{lemma}{Лемма}[section]
\newtheorem{proposition}{Предложение}[section]
\newtheorem{statement}{Утверждение}[section]
\newtheorem{corollary}{Следствие}[section]
\newtheorem{definition}{Определение}[section]


\newcommand{\todoi}[1]{\todo[inline]{#1}}
\def\proof{{\noindent{\bf Доказательство.}} }
\def\A{{\mathfrak{A}}}
\def\K{{\mathcal{K}}}
\def\U{{\mathcal{U}}}
\def\P{{\mathcal{P}}}
\def\F{{\mathcal{F}}}
\def\S{{\mathcal{S}}}
\def\L{{\mathcal{L}}}
\def\C{{\mathcal{C}}}
\def\Z{{\mathbb{Z}}}
\def\N{{\mathbb{N}}}
\def\Q{{\mathbb{Q}}}
\def\Th{{\mathrm{Th}}}
\def\Tha{{\mathrm{Th}_\forall}}
\def\The{{\mathrm{Th}_\exist}}
\def\CG{{\mathrm{CGr}}}
\def\ui{{\mathrm{UI}}}
\def\HP{\textbf{HP}}
\def\JEP{\textbf{JEP}}
\def\AP{\textbf{AP}}
\def\AAP{\textbf{AAP}}


\begin{document}
\maketitle
\tableofcontents
\listoftodos



\section{Введение}

\section{Ультраоднородные группы}

Пусть $\K$ некоторый универсальный класс. Определим некоторые свойства на классе $\K$ согласно книге Ходжеса [???] \todo{Вставить ссылку на Ходжеса}.

\noindent \HP  (\textit{Hereditary property}): Если $A \in \K$ и $B$ кончно порожденная подструктура в $A$, то $B$ изморфна некторой структуре в $\K$. 

\noindent \JEP  (\textit{Join Embeding Property}): Если $A, B \in \K$ тогда существует такая структура $C \in \K$, что $A$ и $B$ вкладываются в $C$.

\noindent \AP  (\textit{Amalgamation Property}): Если $A, B$ и $C$ структуры из $\K$ и $e : A \rightarrow B$ и $f : A \rightarrow C$ --- вложения, тогда существует структура $D \in \K$ и вложения $g : B \rightarrow D$ и $h : C \rightarrow D$ такие, что $ge = hf$. 

Пусть $G$ --- группа, \textit{$p$-высотой} элемента $g \in G$ называется наибольшее число $k$ такое, что в группе $G$ разрешимо уравнение $p^k x = g;$ $p$-высоту элемента $g$ будем обозначать $h_{p,G}(g).$ 

Введем новое свойство \AAP{} путем расширения свойства \AP{} следующим образом:

\noindent \AAP (\textit{Admissible Amalgamation Property}): Если $A, B$ и $C$ структуры из $\K$ и $e : A \rightarrow B$ и $f : A \rightarrow C$ --- вложения такие, что для любого элемента $a \in A$ $h_{p,B}(e(a)) = h_{p,C}(f(a))$. Тогда существует структура $D \in \K$ и вложения $g : B \rightarrow D$ и $h : C \rightarrow D$ такие, что $ge = hf$. 

Пусть $\K$ --- универсальны класс абелевых групп из $\A_p$, $C(\K)$ --- каноническая группа для класса $\K$. Рассмотрим класс всех конечно порожденных подгрупп группы $C(\K)$, обозначим его $FG(C(\K))$. Пусть $A$ и $B$ две конечно порожденные группы из $FG(C(\K))$. Введем оператор $JE(A, B) = C$, где $C$ --- конечно порожденная абелева группа, определяемая следующим образом:
\begin{enumerate}
\item Вычислим инварианты $\ui_p(A)$ и $\ui_p(B)$ для групп $A$ и $B$.
\item По данным инвариантам определим вектор $v$ как покомпонентный максимум из векторов $\ui_p(A)$ и $\ui_p(B)$. Легко понять, что $v$ будет допустимым инвариантом.
\item Группу $C$ определим как каноническую группу для допустимого инварианта $v$.
\end{enumerate}

\begin{lemma}\label{lm:OperatorJE}
В обозначениях выше, если $A$ и $B$ --- конечно порожденные группы из $\K$, тогда $JE(A,B)$ также группа из класса $\K$.
\end{lemma}

\proof Из построения группы $C = JE(A,B)$ следует, что она будет подгруппой канонической группы $C(\K)$ для класса $\K$. Следовательно, $C \in \K$. $\square$

\begin{lemma}
Пусть класс $\K$ порождается своей периодической частью. Тогда для класса $FG(C(\K))$ выполняются свойства \HP, \JEP{} и \AAP.
\end{lemma}

\proof Выполнение свойства \HP{} следует из определения.

Для выполнения свойства \JEP{} для групп $A$ и $B$ в качестве группы $C$ выберем значение оператора $JE(A,B)$. По лемме \ref{lm:OperatorJE} группа $C \in \K$, а то, что группы $A$ и $B$ будут подгруппами в группе $C$ следует из построения группы $C$. 

\todo{Вставить доказательство про \AAP}
$\square$



Будем говорить, что группа $D$ обладает свойством \textit{$h_p$-ультраоднородности}, если любой гомоморфизм сохраняющий высоты между двумя конечными подгруппами $D$ расширяется до автоморфизма группы $D$.

\begin{theorem}[Аналог теоремы Фраиссе]\label{th:Fraisse}
Пусть универсальный класс абелевых групп $\K$ обладает свойствами \HP, \JEP{} и \AAP. Тогда существует единственная с точностью до изоморфизма группа $F_r(\K)$ со свойствами:
\begin{enumerate}
\item $F_r(\K)$ счетная;
\item Любая конечно порожденная группа из $\K$ изоморфна некоторой подгруппе из $F_r(\K)$;
\item $F_r(\K)$ обладает свойством $h_p$-ультраоднородности.
\end{enumerate}
\end{theorem}

\proof \todo{Вставить доказательство.}

В книге [???] группа $F_r(\K)$ называется пределом Фраиссе для класса $FG(\K)$ (отсюда и обозначения ее в этой статье). В других источниках такие пределы называют $P$-ультраоднородными (где $P$ --- фиксированная амальгама из класса систем $\K$).

\begin{corollary}
Если $\K$ универсальный класс для которого существует предел Фраиссе $F$, тогда для класса $\K$ существует и предел $F_r(\K)$, такой, что $F \cong F_r(\K)$.
\end{corollary}

\proof Заметим, что если на классе групп выполнено свойство \AP, то будет выполнено и свойство \AAP. Если существует предел Фраиссе $F$, то он будет удовлетворят всем свойствам из теоремы \ref{th:Fraisse}, и следовательно, по этой же теореме группы $F$ и $F_r(\K)$ будут изоморфны. $\square$




\section{Экзистенциально замкнутые $\K$-группы}

Пусть $\K$ универсальный класс абелевых групп. Группа $A$ называется \textit{экзистенциально замкнутой} относительно класса $\K$, если любая конечная система уравнений и неравенств над группой $A$ совместная с $\Tha(\K)$ имеет решение в группе $A$. Обозначим $\K_{ec}$ класс экзистенциально замкнутых групп из класса $\K$.

\begin{lemma}
Группа $A$ не является экзистенциально замкнутой группой, если выполнено хотя бы одно из следующих условий:
\begin{enumerate}
\item $ucl(A) < \K$;
\item $A \big/ T(A)$ не является делимой группой;
\end{enumerate}
\end{lemma}

\proof \todo{Доказать лемму.}

Пусть $\K$ универсальный класс абелевых групп из $\A_p$. Как и раньше, с помощью универсальных инвариантов $\ui_p(\K)$ класса $\K$ можно построить каноническую $p$-групп $C(\K)$, которая имеет вид:
\begin{equation}\label{eq:CanonnicalGroup}
 C = C^{\aleph_0}(p^a) \oplus T \oplus C^l(p^\infty) \oplus B,
 \end{equation}
где группа $T = \bigoplus\limits_{ a < t \leq b} C^{w_t}(p^t)$, где $w_t = \gamma_{p,t} - \gamma_{p,t+1}$, и группа $B$ либо $\Z$, при $l = 0$ и $\delta = 1$, либо $B = 0$, в остальных случаях.

По канонической группе $C(\K)$ построим вспомогательную группу $C^*(\K)$, которая будет иметь вид (\ref{eq:CanonnicalGroup}) только за одним исключением: первое слагаемое $C(p^a)$ будет входить в разложение в степени $w$, где $w$ --- произвольный кардинал.

Следующая теорема описывает структуру экзистенциально замкнутых групп.
\begin{theorem}\todo{Проверить формулировку теоремы}
Пусть $\K$ универсальный класс, $A \in \K_{ec}$. И пусть $a, b$ и $l$ --- параметры канонической группы $C(\K)$. Тогда:
\begin{enumerate}
\item Если $\K = ucl(T(\K))$ и $\delta(\K) = 1$, тогда
\begin{itemize}
\item Если $l = \infty$, то $A$ --- делимая группа из $\A_p$;
\item Если $l \neq \infty$, то
$$A = C^*(\K) \oplus (\Q^+)^w.$$
\end{itemize}
\item Если $T(\K)$ --- ограниченный класс и $\delta(\K) = 0$, тогда
$$A = C^*(\K).$$
\item Если $T(\K)$ --- ограниченный класс и $\delta(\K) = 1$, тогда
$$A = C^*(\K) \oplus (\Q^+)^w.$$
\end{enumerate}
\end{theorem}

\proof \todo{Доказать теорему.}

\todo{Вставить определение модельного компаньона.}
\begin{theorem}
Для любого универсального класса $\K$ из $\A_p$, класс $\K_{ec}$ аксиоматизируем, и следовательно, $\Tha(\K)$ имеет модельный компаньон ($\Th(\K_{ec})$).
\end{theorem}

\proof \todo{Доказать.}




 
\section{Простые модели $\Th(\K_{EC})$}

Пусть $T$ --- непротиворечивая теория языка $L$. Модель $\mathcal{M}$ теории $T$ называется \textit{простой} если для любой модели $\mathcal{N}$ теории $T$ существует элементарное вложение $e : \mathcal{M} \rightarrow \mathcal{N}$. Вложение $e : \mathcal{M} \rightarrow \mathcal{N}$ двух моделей теории $T$ называется \textit{элементарным}, если для любого предложения $\varphi$ теории $T$ с константами из $\mathcal{M}$ верно $$\mathcal{M} \models \varphi \Leftrightarrow \mathcal{N} \models \varphi.$$


\begin{theorem}
Для любого универсального класса $\K$ абелевых групп из $\A_p$ для теории $\Th(\K_{EC})$ существует простая модель. Кроме того:
\begin{enumerate}
\item Если $\K$ --- ограниченный класс, то $C(\K)$ --- простая модель для теории $\Th(\K_{EC}).$
\item Если $\K$ --- универсальный класс абелевых групп без кручения, то $\Q^+$ --- простая модель для $\Th(\K_{EC}).$
\item Если $T_p(\K) \neq 0$ и $\delta(\K) = 1$, то простая модель изоморфна $C(\K) \oplus \Q^+.$
\end{enumerate}
\end{theorem}

\end{document}